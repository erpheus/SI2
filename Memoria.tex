\documentclass[a4paper, 10pt]{article}
\usepackage[utf8]{inputenc}
\usepackage[spanish]{babel}
\usepackage{hyperref}
\usepackage{tabularx}
\usepackage{inconsolata}
\usepackage{enumitem}
\usepackage{graphicx}
\usepackage{subfigure}
\usepackage{listings}
\usepackage{color}
\usepackage{appendix}

\definecolor{mygreen}{rgb}{0,0.6,0}
\definecolor{mygray}{rgb}{0.5,0.5,0.5}
\definecolor{mymauve}{rgb}{0.58,0,0.82}

\lstset{ %
	backgroundcolor=\color{white},   % choose the background color; you must add \usepackage{color} or \usepackage{xcolor}
	basicstyle=\footnotesize,        % the size of the fonts that are used for the code
	breakatwhitespace=false,         % sets if automatic breaks should only happen at whitespace
	breaklines=true,                 % sets automatic line breaking
	captionpos=b,                    % sets the caption-position to bottom
	commentstyle=\color{mygreen},    % comment style
	deletekeywords={...},            % if you want to delete keywords from the given language
	escapeinside={\%*}{*)},          % if you want to add LaTeX within your code
	extendedchars=true,              % lets you use non-ASCII characters; for 8-bits encodings only, does not work with UTF-8
	frame=single,                    % adds a frame around the code
	keepspaces=true,                 % keeps spaces in text, useful for keeping indentation of code (possibly needs columns=flexible)
	keywordstyle=\color{blue},       % keyword style
	language=Octave,                 % the language of the code
	morekeywords={*,...},            % if you want to add more keywords to the set
	numbers=left,                    % where to put the line-numbers; possible values are (none, left, right)
	numbersep=5pt,                   % how far the line-numbers are from the code
	numberstyle=\tiny\color{mygray}, % the style that is used for the line-numbers
	rulecolor=\color{black},         % if not set, the frame-color may be changed on line-breaks within not-black text (e.g. comments (green here))
	showspaces=false,                % show spaces everywhere adding particular underscores; it overrides 'showstringspaces'
	showstringspaces=false,          % underline spaces within strings only
	showtabs=false,                  % show tabs within strings adding particular underscores
	stepnumber=2,                    % the step between two line-numbers. If it's 1, each line will be numbered
	stringstyle=\color{mymauve},     % string literal style
	tabsize=2,                       % sets default tabsize to 2 spaces
	title=\lstname                   % show the filename of files included with \lstinputlisting; also try caption instead of title
}

\makeatletter
\def\@seccntformat#1{%
  \expandafter\ifx\csname c@#1\endcsname\c@section\else
  \csname the#1\endcsname\quad
  \fi}
\makeatother

\newcommand{\tabitem}{\vspace{1mm}~~\llap{\textbullet}~~}

\hypersetup{
    colorlinks=true,
    citecolor=black,
    linkcolor=black,
    urlcolor=blue
}

%% Titulo y autores
\title{Sistemas Informáticos 2\\Práctica 1b}
\author{Cristina Kasner Tourné\and Guillermo Guridi Mateos}
\date{\today}

%% Documento
\begin{document}
\maketitle
\newpage

\section{Cuestión 1}\textit{\textbf{VisaDAOLocal.java}}


En este archivo se especifican los métodos que va a utilizar la interfaz \texttt{VisaDAOLocal}.


Vemos que tiene la etiqueta @Local , que significa:
\begin{itemize}
	\item Debe ejecutarse en la misma Maquina Virtual Java del EJB al que accede
	\item Puede ser un componente web u otro EJB
	\item Para el cliente local, la localización del EJB al que accede no es transparente
\end{itemize}

\section{Ejercicio 1}

Para ajustar los métodos a la interfaz hemos quitado la etiqueta de \texttt{synchronize} ya que no se utilizan en los métodos especificados en \texttt{VisaDAOLocal}.

También hemos cambiado el retorno del método \texttt{getPagos}, antes devolvía \textbf{ArrayList} y ahora devuelve \textbf{PagoBean []} para respetar las declaraciones de \texttt{VisaDAOLocal}.

\section{Ejercicio 2}
Hemos eliminado de los \textit{servlets} todo lo referente a los \textit{Web Services} (que estaban en los respectivos métodos \texttt{ProcessRequest})ya que ahora hacemos uso del EJB que hace uso de la interfaz \texttt{VisaDAOLocal}.

En \texttt{getPagos} hemos vuelto a cambiar el tipo de retorno para que coincida con la declaración que hay en \texttt{VisaDAOLocal}.
\section{Cuestión 2}
Vemos que en el archivo \textit{application.xml} hay dos módulos:
\begin{itemize}
	\item Uno que contiene los EJBs , y en general el servidor.
	\item En el otro se declara la parte web de la aplicación indicando el \texttt{.war} del cliente , y la dirección en la que será desplegada (\texttt{/P1-ejb-cliente})
 \end{itemize}
 
 Cuando ejecutamos el comando \texttt{jar -tvf} vemos que nos muestra la tabla de contenidos del archivo al que se lo hemos aplicado.
 
 Vamos a explicar el caso del \texttt{.ear} .
 
 Contiene 4 archivos:
 \begin{itemize}
 	\item \textbf{application.xml}, del que ya hemos hablado
 	\item \textbf{MANIFEST.MF} , que contiene información sobre la versión de \texttt{ant} usada para la empaquetación.
 	\item \textbf{P1-ejb-cliente.war}, que es el cliente empaquetado.
 	\item \textbf{P1-ejb.jar} , que es el servidor empaquetado.
 \end{itemize} 
 
 \section{Ejercicio 3}
 
 \begin{figure}[hbtp]
 	\centering
 	\includegraphics[width=0.8\textwidth]{}
 	\caption{Comprobar en la administración de Glassfish el despliegue de la aplicación}
 \end{figure}
 
 \section{Ejercicio 5}
 \begin{itemize}
 	\item Modificar \texttt{ TarjetaBean.java}
 	
 	
 	 Además de añadir saldo como atributo hemos añadido los métodos getSaldo y setSaldo.
 	 
 	 
 	 \item Modificar \texttt{ VisaDAOBean.java}
 	 
 	 
 	 Hemos creado las dos consultas y hemos modificado el método \texttt{RealizaPago} para que ejecute las dos consultas anteriores usando los \textit{prepared statements}.
 	 
 	 Al contrario que las otras consultas que se ejecutan en \texttt{Realizapago} , estas no dan la opción de no usar el \textit{prepared}.
 	 
 	 
 	 Para la parte de lanzar la excepción en caso de que algo falle, hemos un String que explica la razón por la que ha fallado y se los mandamos al cliente junto con la \texttt{EJBException}.
 	 
 	 \item Modificar \texttt{ProcesaPago}
 	 
 	 Hemos capturado la excepción con \texttt{try/catch}
 \end{itemize}
 
 
\section{Ejercicio 6}
	\begin{figure}[htbp]
		\centering
		
		\subfigure[Saldo ANTES del pago]{\includegraphics[width=55mm]{}}
		\subfigure[Realización del pago]{\includegraphics[width=55mm]{}}\vspace{3mm} \hspace{10mm}
		\subfigure[Saldo DESPUÉS del pago]{\includegraphics[width=80mm]{}}
		\caption{Realización de pago y comprobación de saldo}
	\end{figure}
 
 
 
\newpage
\appendix
\section{Apéncices}

\end{document}